\documentclass[11pt,a4paper]{article}

\usepackage[utf8]{inputenc}
\usepackage[T1]{fontenc}
\usepackage[spanish]{babel}
\usepackage{geometry}
\geometry{margin=2.5cm}
\usepackage{hyperref}
\usepackage{amsmath, amssymb}
\usepackage{booktabs}
\usepackage{listings}
\usepackage{xcolor}
\usepackage{verbatim}
\usepackage{graphicx}
\usepackage{fancyhdr}

\pagestyle{fancy}
\fancyhf{}
\lhead{Computación, Algoritmos y Estructuras de Datos - FIUBA}
\rhead{TP3 - Needleman--Wunsch}
\cfoot{\thepage}

\lstset{
  basicstyle=\ttfamily\small,
  breaklines=true,
  frame=single,
  columns=fullflexible
}



\begin{document}

\begin{titlepage}
    \centering

    % Logo (ajustá ruta y tamaño)
    \includegraphics[width=0.45\textwidth]{uba.png}\\[1cm]

    {\Large \bfseries Universidad de Buenos Aires (FIUBA)}\\[0.2cm]
    {\large Maestría en Inteligencia Artificial}\\[1cm]

    \rule{\textwidth}{1.2pt}\\[0.4cm]
    {\Huge \bfseries Trabajo Práctico 3}\\[0.2cm]
    {\LARGE \bfseries Needleman--Wunsch (Alineación Global)}\\[0.2cm]
    \rule{\textwidth}{1.2pt}\\[1.2cm]

    \begin{tabular}{rl}
        \textbf{Materia:} & Computación, Algoritmos y Estructuras de Datos \\
        \textbf{Docente:} & Dr. Lic. Camilo Argoty \\
        \textbf{Alumna:} & Esp. Lic. Noelia Qualindi \\
        \textbf{SIU:} & a1411 \\
    \end{tabular}\\[10cm]  % Espacio después de la tabla

    % Repositorio abajo
    {\large Repositorio del trabajo:\\[0.2cm]}
    \url{https://github.com/noequalindi/computing_algorithms/tree/main/tp3}\\[2cm]  % Espacio más grande antes de finalizar el título
    
\end{titlepage}

\section{Parte 1: Conceptos teóricos}

\subsection{Secuencias de nucleótidos}
Una secuencia de nucleótidos es una cadena sobre un alfabeto finito, típicamente $\{A,C,G,T\}$ para ADN (o $\{A,C,G,U\}$ para ARN). Compararlas permite identificar similitudes, mutaciones (sustituciones, inserciones, deleciones) y relaciones evolutivas.

\subsection{Alineación de secuencias}
Un alineamiento inserta huecos (gaps) ``-'' para poner en correspondencia posiciones entre dos secuencias.
\begin{itemize}
  \item \textbf{Alineación global}: alinea de extremo a extremo (se fuerza cubrir toda la longitud).
  \item \textbf{Alineación local}: busca el segmento más similar (no necesariamente usa toda la secuencia).
\end{itemize}
En un alineamiento aparecen:
\begin{itemize}
  \item \emph{matches}: misma letra,
  \item \emph{mismatches}: letras distintas,
  \item \emph{gaps}: inserciones/deleciones modeladas con ``-''.
\end{itemize}

\subsection{Modelo de puntuación}
Se usa un esquema simple:
\begin{itemize}
  \item Match: $+1$
  \item Mismatch: $-1$
  \item Gap: $-2$
\end{itemize}
(En modelos más realistas, se distinguen penalizaciones de apertura/extensión de gap; aquí se usa penalización constante por gap.)

\subsection{Algoritmo de Needleman--Wunsch}
Needleman--Wunsch resuelve alineación global mediante programación dinámica:
\begin{itemize}
  \item Construye una matriz $F$ de tamaño $(n+1)\times(m+1)$.
  \item Inicializa primera fila/columna acumulando gaps.
  \item Usa la recurrencia:
  \[
  F[i,j]=\max\begin{cases}
  F[i-1,j-1] + s(x_i,y_j) \\
  F[i-1,j] + \text{gap} \\
  F[i,j-1] + \text{gap}
  \end{cases}
  \]
  \item Luego aplica \emph{traceback} desde $F[n,m]$ para recuperar un alineamiento óptimo.
\end{itemize}
El puntaje óptimo de alineación global queda en la celda inferior derecha $F[n,m]$.

\subsection{Referencia}
Needleman, S. B., \& Wunsch, C. D. (1970). \emph{A general method applicable to the search for similarities in the amino acid sequence of two proteins}. Journal of Molecular Biology, 48(3), 443--453.

\section{Parte 2: Implementación}
Se implementó desde cero Needleman--Wunsch en Python con el esquema de puntuación indicado. Para cada par de secuencias el programa imprime:
\begin{itemize}
  \item matriz completa de puntuación,
  \item un alineamiento global óptimo,
  \item puntaje final.
\end{itemize}

\subsection{Cómo correr}
\begin{lstlisting}
cd tp3
python tp3_needleman_wunsch.py --examples
\end{lstlisting}

\section{Evidencia de ejecución (salidas del programa)}

\subsection{Ejecución con tres parejas}
A continuación se muestra la salida obtenida al ejecutar el comando \texttt{--examples}. Se incluyen las matrices completas, los alineamientos óptimos y el puntaje final de cada caso.
\newpage
\begin{verbatim}
python tp3_needleman_wunsch.py --examples

============================================================
Sequence 1: GATTACA
Sequence 2: GCATGCU
Score matrix:
      _   G   C   A   T   G   C   U
  _   0  -2  -4  -6  -8 -10 -12 -14
  G  -2   1  -1  -3  -5  -7  -9 -11
  A  -4  -1   0   0  -2  -4  -6  -8
  T  -6  -3  -2  -1   1  -1  -3  -5
  T  -8  -5  -4  -3   0   0  -2  -4
  A -10  -7  -6  -3  -2  -1  -1  -3
  C -12  -9  -6  -5  -4  -3   0  -2
  A -14 -11  -8  -5  -6  -5  -2  -1

Optimal alignment (global):
GATTACA
GCATGCU
Total score: -1
============================================================

============================================================
Sequence 1: ACGT
Sequence 2: ACCT
Score matrix:
    _  A  C  C  T
 _  0 -2 -4 -6 -8
 A -2  1 -1 -3 -5
 C -4 -1  2  0 -2
 G -6 -3  0  1 -1
 T -8 -5 -2 -1  2

Optimal alignment (global):
ACGT
ACCT
Total score: 2
============================================================

============================================================
Sequence 1: ATGCT
Sequence 2: AGCT
Score matrix:
      _   A   G   C   T
  _   0  -2  -4  -6  -8
  A  -2   1  -1  -3  -5
  T  -4  -1   0  -2  -2
  G  -6  -3   0  -1  -3
  C  -8  -5  -2   1  -1
  T -10  -7  -4  -1   2

Optimal alignment (global):
ATGCT
A-GCT
Total score: 2
============================================================
\end{verbatim}

\subsection{Verificación del puntaje del Caso 1}
En el Caso 1 se obtuvo un alineamiento sin gaps:
\[
\texttt{GATTACA}\quad\text{vs}\quad\texttt{GCATGCU}.
\]
Con match $=+1$ y mismatch $=-1$, el puntaje total es:
\[
(+1) + (-1) + (-1) + (+1) + (-1) + (+1) + (-1) = -1,
\]
lo cual coincide con el valor reportado por el algoritmo y con la celda $F[n,m]$ de la matriz de puntuación.

\subsection{Repositorio}
El código fuente y salidas reproducibles se entregan vía repositorio:
\begin{center}
\url{https://github.com/noequalindi/computing_algorithms/tree/main/tp3}
\end{center}

\section{Bibliografía}
\begin{thebibliography}{9}

\bibitem{nw1970}
S. B. Needleman and C. D. Wunsch.
A general method applicable to the search for similarities in the amino acid sequence of two proteins.
\textit{Journal of Molecular Biology}, 48(3):443--453, 1970.

\bibitem{durbin}
Richard Durbin, Sean Eddy, Anders Krogh, Graeme Mitchison.
\textit{Biological Sequence Analysis: Probabilistic Models of Proteins and Nucleic Acids}.
Cambridge University Press, 1998.
(Algoritmos de alineación y programación dinámica).

\end{thebibliography}

\end{document}
