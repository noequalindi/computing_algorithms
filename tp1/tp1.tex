\documentclass[11pt,a4paper]{article}

\usepackage[utf8]{inputenc}
\usepackage[T1]{fontenc}
\usepackage[spanish]{babel}
\usepackage{geometry}
\geometry{margin=2.5cm}
\usepackage{hyperref}
\usepackage{amsmath, amssymb}
\usepackage{booktabs}
\usepackage{listings}
\usepackage{xcolor}
\usepackage{graphicx}

\lstset{
  basicstyle=\ttfamily\small,
  breaklines=true,
  frame=single,
  columns=fullflexible
}

\title{TP1 - Máquina de Turing: Suma Binaria}
\author{Noelia Melina Qualindi \\ SIU: a1411}
\date{\today}

\begin{document}
\maketitle

\section{Enunciado elegido}
Según el último dígito del SIU (a1411), corresponde el punto 1 del TP1: \emph{suma de dos números en binario}.

\section{Formato y supuestos de entrada}
Se implementa una Máquina de Turing (MT) de 2 cintas (multi-tape) definida en un archivo YAML:
\begin{itemize}
  \item Cinta 1: número binario $A$
  \item Cinta 2: número binario $B$ (también se usa para escribir el resultado)
\end{itemize}

La MT opera en \emph{little-endian} (bit menos significativo primero). Por eso, el runner invierte las cadenas ingresadas en binario normal (big-endian) antes de cargar la cinta, y vuelve a invertir el resultado al finalizar.

\section{Idea de la máquina}
La suma binaria se modela con dos ``modos'' (estados):
\begin{itemize}
  \item \textbf{no carry}: no hay acarreo.
  \item \textbf{carry}: hay acarreo.
\end{itemize}
La máquina recorre los bits desde el menos significativo hacia la izquierda, escribiendo el bit suma en la cinta 2 y actualizando el estado según el acarreo.

\section{Archivos entregados}
\begin{itemize}
  \item \texttt{tp1/binary\_addition.yaml}: definición de la MT.
  \item \texttt{tp1/tp1\_run\_tm.py}: runner para ejecutar y producir evidencia.
\end{itemize}

\section{Cómo correr (evidencia)}
\begin{lstlisting}
cd tp1
python tp1_run_tm.py --a 1011 --b 111
\end{lstlisting}
La salida incluye:
\begin{itemize}
  \item resultado en binario (big-endian),
  \item tapes finales (útil para captura),
  \item cantidad de pasos.
\end{itemize}

\section{Evidencia de ejecución (local)}
Se ejecutó el script \texttt{tp1\_run\_tm.py} con tres casos de prueba.

\subsection{Caso 1: 1 + 1}
\begin{verbatim}
python tp1_run_tm.py --a 1 --b 1

Entrada A (bin): 1
Entrada B (bin): 1
Resultado (bin): 10
Tape1 final (trim): 1
Tape2 final (trim): 10
\end{verbatim}

\subsection{Caso 2: 1111 + 1}
\begin{verbatim}
python tp1_run_tm.py --a 1111 --b 1

Entrada A (bin): 1111
Entrada B (bin): 1
Resultado (bin): 10000
Tape1 final (trim): 1111
Tape2 final (trim): 10000
\end{verbatim}

\subsection{Caso 3: 0 + 0}
\begin{verbatim}
python tp1_run_tm.py --a 0 --b 0

Entrada A (bin): 0
Entrada B (bin): 0
Resultado (bin): 0
Tape1 final (trim): 0
Tape2 final (trim): 0
\end{verbatim}
\subsection{Caso 4: 1101 + 101}
En el simulador en línea este caso se ingresa como \texttt{1101\#101}. En la ejecución local se pasan los operandos por parámetros separados.

\begin{verbatim}
python tp1_run_tm.py --a 1101 --b 101
============================================================
Entrada A (bin): 1101
Entrada B (bin): 101
Resultado (bin): 10010
Pasos: 10
Tape1 final (trim): 1101
Tape2 final (trim): 10010
============================================================
\end{verbatim}


\section{Comprobación en simulador en línea}

La máquina de Turing fue validada en un simulador web. Se adjunta evidencia visual de la ejecución y un enlace compartible que permite reproducir la comprobación.

\subsection{Simulador y enlace}
\begin{itemize}
  \item Simulador: \url{https://turingmachinesimulator.com/}
  \item Enlace (Share Link) de la máquina validada: \url{http://turingmachinesimulator.com/shared/yoxceanwzp}
\end{itemize}

\paragraph{Caso 1101\#101}: el siguiente caso se utilizó tanto para la validación en el simulador en línea como para la ejecución local.

\begin{itemize}
  \item Entrada en simulador online: \texttt{1101\#101}
  \item Entrada en ejecución local: \texttt{--a 1101 --b 101}
  \item Salida obtenida: \texttt{10010}
\end{itemize}

\begin{figure}[h!]
  \centering
  \includegraphics[width=0.95\textwidth]{online_turing.png}
  \caption{Comprobación en simulador en línea de la suma binaria con entrada \texttt{1101\#101}.}
  \label{fig:tp1-online-1101-101}
\end{figure}

En la Figura~\ref{fig:tp1-online-1101-101} se observa el estado final de la ejecución,
donde la cinta de salida contiene el resultado correcto de la suma.


\section{Repositorio}
El código fuente y evidencia se entregan vía repositorio:
\begin{center}
\url{https://github.com/noequalindi/computing_algorithms/tree/main/tp1}
\end{center}

\end{document}
